\documentclass[11pt,wide]{article}

\usepackage[T1]{fontenc}
\usepackage[utf8]{inputenc}
\usepackage{polski}
\usepackage{lmodern}
\usepackage{amsmath,amsthm}
\usepackage{amsfonts}
\usepackage{algpseudocode}
\usepackage{pgfplots}
\usepackage{enumerate}
\newtheorem{thr}{Twierdzenie}
\newtheorem{lem}[thr]{Lemat}
\newtheorem*{dfn}{Definicja}
\usepackage{graphicx}
\graphicspath{ {img/} }
\usepackage[export]{adjustbox}
\title{Analiza numeryczna(M) - Pracownia 3 - Zadanie 17}
\date{Styczeń 29, 2017}
\author{Wojciech Balik}


\begin{document}

\maketitle                
\thispagestyle{empty}     
\tableofcontents          
\section{Wstęp}
W praktycznych zastosowaniach algebry liniowej dość popularny jest rozkład LU, gdzie dla danej macierzy $A \in \mathbb{R}^{n \times n}$ szukane są macierze $L \in \mathbb{L}^{n \times n}$ oraz $U \in \mathbb{U}^{n \times n}$ spełniające równanie:
\begin{center}
$A = LU$
\end{center}
Rozkład który zostanie omówiony jest szczególnym przypadkiem rozkładu LU. Rozkład Choleskiego, gdzie o macierzy $A$ zakładamy że jest symetryczna oraz dodatnio określona, jest postaci:
\begin{center}
$A = LL^T$
\end{center}


\section{Algorytm Banachiewicza-Choleskiego}
Załóżmy że mamy daną macierz $M \in \mathbb{R}^{n \times n}$ symetryczną oraz dodatnio określoną.Chcemy znaleźć rozkład $M$ w postaci:
\begin{equation}
M = LL^T
\end{equation}
Gdzie L jest macierzą dolnotrójkątną stopnia $n$.\newline
\noindent
Zauważmy że korzystając z symetryczności $M$ możemy rozpisać to równanie jako:

\[
 \left[
        \begin{array}{cccc}
         	a_{1,1} & a_{2,1} & \ldots & a_{n,1}\\
  			a_{2,1} & a_{2,2} & \ldots & a_{n,2} \\
  			\vdots &  & & \vdots\\
  			a_{n,1} & a_{n,2} & \ldots & a_{n,n} 
         \end{array}
    \right]
      \quad
 = \quad \left[
       \begin{array}{cccc}
          l_{1,1} & 0 & \ldots & 0\\
          l_{2,1} & l_{2,2} & & 0 \\
          \vdots &  & \ddots & \vdots\\ 
          l_{n,1} & l_{n,2} & \ldots & l_{n,n}
       \end{array}
     \right]
	 \left[
       \begin{array}{cccc}
          l_{1,1} & l_{2,1} & \ldots & l_{n,1}\\
          0 & l_{2,2} &  & l_{n,2} \\
          \vdots &  & \ddots & \vdots\\ 
          0 & \ldots & 0 & l_{n,n}
       \end{array}
     \right]
\]
\noindent
Wówczas możemy wyznaczyć elemeny macierzy $L$ idąc wiersz po wierszu, otrzymując:
\begin{center}
$l_{1,1}^2 = a_{1,1} \Rightarrow l_{1,1} = \sqrt{a_{1,1}}$ \\
$ l_{2,1}l_{1,1} = a_{2,1} \Rightarrow l_{2,1} = \frac{a_{2,1}}{l_{1,1}}$ \\
$\vdots$ 
\end{center}
\begin{equation}
l_{i,i} = \sqrt{a_{i,i} - \displaystyle\sum_{k=1}^{i-1}l_{i,k}^2}
\end{equation}
\begin{equation}
l_{i,j} = \frac{1}{l_{j,j}} (a_{i,j} - \displaystyle\sum_{k=1}^{j-1} l_{i,k}l_{j,k})
\end{equation}
Dzięki dodatniej określoności macierzy $M$, wszystkie wartości pod pierwiastkiem we wzorze (2) są dodatnie, więc nie ma potrzeby używać liczb zespolonych(szczegóły można doczytać w \cite{kincaid}).\newline
Koszt wyznaczenia takiego rozkładu to $O(n^2)$.Zauważmy również że nie potrzebujemy trzymać w pamięci komputera dwóch macierzy $L$ oraz $L^T$, co pozwala zaoszczędzić nieco miejsca.



\section{Zastosowania Rozkładu Choleskiego}
Najbardziej oczywistym zastosowaniem jest rozwiązywanie układów równań.Załóżmy że mamy układ równań:
\begin{equation}
Ax = b
\end{equation}
Gdzie $A$ posiada rozkład Choleskiego.Wówczas otrzymujemy:
\begin{equation}
LL^Tx = b
\end{equation}
Zauważmy że rozwiązując równanie:
\begin{equation}
Ly = b
\end{equation}
oraz podstawiając do (3) a następnie mnożąc obustronnie przez $L^{-1}$, otrzymujemy:
\begin{equation}
L^Tx = y
\end{equation}
Rozwiązanie równań (4) oraz (5) to koszt $O(n^2)$, gdzie dla porównania, zwykła eliminacja Gaussa kosztuje $O(n^3)$.
\noindent
Mogłoby się wydawać że powyższa metoda nie znajdzie swojego zastosowania przez to że wymagane jest aby macierz A była symetryczna oraz dodatnio określona. Okazuje się jednak że takie macierze występują we wielu problemach, m.in. w równaniach różniczkowych cząstkowych, statystyce.

\subsection{Problem najmniejszych kwadratów}
Załóżmy że mamy układ równań:
\begin{equation}
Ax = b
\end{equation}
Gdzie $A \in \mathbb{R}^{m \times n}, \quad m > n,\quad rk(A) = n$.\newline
Taki układ zazwyczaj nie ma rozwiązań, ale można znaleźć wektor x który minimalizuje wartość:
\begin{center}
$\lVert Ax-b\rVert_2^2$
\end{center}
Okazuje się że wektor $x$ musi spełniać:
\begin{equation}
(A^TA)x = A^Tb
\end{equation}
Zauważmy że:
\begin{equation}
(A^TA)^T = A^TA
\end{equation}
oraz dla dowolnego wektora $u \in \mathbb{R}^n$
\begin{equation}
u^TA^TAu = (Au)^T(Au) = \lVert Au \rVert_2^2 > 0
\end{equation}
Macierz jest więc symetryczna oraz dodatnio określona co umożliwia nam użycie rozkładu Choleskiego do rozwiązania tego problemu.

\section{Testy numeryczne}
Testy zostały przeprowadzone w następujący sposób:\newline
Wybrana zostaje macierz $A$ oraz wektor $x$ który będzie prawdziwym rozwiązaniem, obliczany zostaje wektor b przez przemnożenie $M$ oraz $x$. Następnie wyznaczany zostaje wektor $\tilde{x}$ za pomocą rozkładu Choleskiego.\newline
Do prezentacji błędów numerycznych posłużymy się błędem względnym:
\begin{center}
$\frac{\lVert x - \tilde{x} \rVert}{\lVert x \rVert}$
\end{center}
oraz residuum:
\begin{center}
$\lVert b - A\tilde{x} \rVert$
\end{center}

\subsection{Macierz Hilberta}
Macierzą Hilberta nazywamy macierz $H = [h_{ij}]$, gdzie $h_{ij} = \frac{1}{i+j-1}$. Cechą charakterystyczną tej macierzy jest to że jest bardzo źle uwarunkowana - $cond(H_n) = O(\frac{e^{3.5n}}{\sqrt{n}})$. Używanie jej w obliczeniach numerycznych, nawet dla niewielkich n może spowodować duże błędy numeryczne.\newline
Poniższa tabelka prezentuje błąd względny oraz residuum, gdzie wektor $b$ został tak dobrany że $x = [1, 2, ..., n]^T$

\begin{center}
\begin{tabular}{|c|c|c|}
  \hline 
  	$n$ & $\lVert b - Hx\rVert$ & $\frac{\lVert x - \tilde{x} \rVert}{\lVert x \rVert}$  \\ \hline
  	3 & 0.000e+00 & 1.280e-14\\ \hline
	6 & 4.441e-16 & 1.759e-10\\ \hline
	9 & 1.662e-15 & 4.366e-06\\ \hline
	11 & 2.391e-15 & 7.929e-03\\ \hline

\end{tabular}
\end{center}
\noindent
Problemy związane z błędami numerycznymi wcale nie muszą dotyczyć tylko wyniku.Mimo że macierz Hilberta jest dodatnio określona dla każdego $n$, to dla $n \geq 12$ wskutek zaokrągleń numerycznych wyrażenie pod pierwiastkiem we wzorze (2) jest ujemne. Wniosek z tego płynący jest taki że nawet jeśli macierz jest dodatnio określona w sensie matematycznym, wcale nie oznacza że w sensie numerycznym również tak będzie.

\subsection{Macierz Pei}
Macierz Pei to macierz postaci: $\textbf{1} + \epsilon I_n$, gdzie $\textbf{1}$ to macierz kwadratowa $n \times n$ mająca jedynki na wszystkich współrzędnych.Gdy $\epsilon \approx 0$ macierz jest źle uwarunkowana.Wektor $b$ został dobrany tak aby $x = [1, 1, ..., 1]^T$, oraz $\epsilon = 10e-15$

\begin{center}
\begin{tabular}{|c|c|c|}
  \hline 
  	$n$ & $\lVert b - Px\rVert$ & $\frac{\lVert x - \tilde{x} \rVert}{\lVert x \rVert}$  \\ \hline
  	3 & 4.441e-16 & 1.549e-01\\ \hline
	6 & 1.986e-15 & 2.125e-01\\ \hline
	9 & 3.553e-15 & 8.231e-01\\ \hline
	12 & 7.324e-15 & 1.046e+00\\ \hline
	15 & 3.077e-15 & 1.209e+00\\ \hline
	18 & 3.553e-15 & 1.216e+00\\ \hline

\end{tabular}
\end{center}

\section{Wnioski}
Rozkład Choleskiego zdecydowanie nie nadaje się do rozwiązań ogólnych, gdzie o rozkładanych macierzach nie wiemy nic, ponieważ szansa że taka macierz będzie symetryczna oraz dodatnio określona jest bardzo mała. Rozkład ten znalazł jednak zastosowanie w pewnych szczególnych problemach, sprawdzając się całkiem dobrze. Dość dużym minusem jest to że macierze które formalnie są określone dodatnio, z punktu widzenia obliczeń numerycznych wcale takie nie muszą być, co zmusza nas do użycia innego rozkładu.
\begin{thebibliography}{10}
\bibitem{kincaid}
  David Kincaid, Ward Cheney,
  przekł. Stefan Paszkowski,
  \emph{Analiza Numeryczna},
  Warszawa, WNT, 2006.
\end{thebibliography}


\end{document}