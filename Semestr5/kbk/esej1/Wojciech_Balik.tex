\documentclass[polish]{kbk}
\usepackage{listings}
\usepackage{xcolor}

\begin{document}

\author{Wojciech Balik}
\title{Buffer Overflow vs Ochrona Pamięci}


\maketitle

\begin{abstract}
Niskopoziomowe błędy w oprogramowaniu, a w szczególności błąd przepełnienia
bufora istnieją od bardzo dawna, a jednym z powodów ich powstawania, jest 
obdarzenie programistów zbyt dużym zaufaniem co do kontroli nad pamięcią. 
Z tego powodu powstało wiele mechanizmów ochrony pamięci, mające na celu
wykluczenie pewnych wektorów ataku. Nikomu nie udało się jednak wymyślić
metody która sprawiłaby, że tego typu ataki staną sie niemożliwe, co
potwierdza fakt że techniki takie jak ROP, ret2libc nadal są wykorzystywane.
Błędy takie jak buffer overflow wymagają dość dużej wiedzy by je wykorzystać,
ale za to skutki tego mogą być bardzo poważne. Z tego powodu, mechanizmem 
zapobiagania ich powstawaniu jest odebranie programiście możliwości
szczegółowego zarządzania pamięcią. Jak się okazuje, nie rozwiązuje
to problemu, dlatego że niektóre elementy oprogramowania z natury potrzebują
większych możliwości.
\end{abstract}

\section{Czym jest Buffer Overflow?}
Niektóre języki programowania, takie jak np. C czy C++, oferują programiście
dość dużą kontrolę nad pamięcią. Niestety, jak mówi znane powiedzenie,
``\emph{Z wielką władzą wiąże się wielka odpowiedzialność}'', co w przypadku 
kontroli nad pamięcią znaczy że język gwarantuje programiście wiele możliwości
strzelenia sobie w stopę. Jednym ze skutków takiego podejścia jest podatność 
polegająca na przepełnieniu bufora, to znaczy, próby zapisania do niego 
większej ilości bajtów niż jego wielkość. Prowadzi to do nadpisania danych 
bezpośrednio za buforem, często na tyle wrażliwych by doszło do zdalnego 
wykonania kodu. 
\par
Klasycznym przykładem tego typu podatności jest użycie funkcji gets, co widać na 
poniższym kodzie źródłowym:

\begin{flushleft}
char buf[32];\newline
gets(buf);\newline
puts(buf);
\end{flushleft}

Funkcja gets w żaden sposób nie kontroluje tego, jak dużo bajtów zapisze do bufora, 
co prowadzi do tego, że atakujący ma pełną kontrolę nad np. adresem powrotu 
odłożonym na stosie. Jest to główny powód, dla którego po wydaniu polecenia 
\textbf{man 3 gets} w powłoce Linuxowej, lub nawet przy kompilacji, użytkownik 
jest ostrzegany, by nie używać tej funkcji. Załóżmy że używamy architektury 
64-bitowej, oraz znamy adres pod którym znajduje się bufor, wtedy atak mógłby
być przeprowadzony w następujący sposób:

\begin{enumerate}
\item Wpisujemy 40 bajtów paddingu(32 na bufor + 8 na odłożony na stosie rejestr rbp)
\item Adres powrotu nadpisujemy adresem pozostałej części bufora na stosie(tzn. 
adresem odłożenego na stosie rejestru rsp plus 8)
\item resztę pamięci nadpisujemy naszym kodem
\end{enumerate}

Wtedy po powrocie z funkcji, zamiast skoczyć do kodu który ją wywołał, zostanie
wykonany skok do wstrzykniętego kodu na stosie.

\par

Czasami może zdarzyć się tak, że znamy adres bufora, ale tylko 
w przybliżeniu, tzn. prawdziwy adres jest oddalony o pewną(stosunkowo niewielką)
stałą od tego adresu który znamy. Wówczas efektywną techniką jest dodanie do 
naszego kodu dużego prefiksu składającego się z instrukcji NOP, czyli instrukcji
która nie robi nic(często występujące określenie to \textbf{NOP sled}). 
Wtedy ``trafiając'' naszym znanym adresem w ten prefiks, mamy pewność, że reszta 
kodu się wykona, co znacznie zwiększa prawdopodobieństwo trafienia. 


\section{Zabezpieczenia oraz ich łamanie}
Mimo że podatności typu buffer overflow są błędami programistycznymi, dobrze 
byłoby zapobiegać wykorzystywaniu tego typu błędów. Z tego powodu 
wprowadzono wiele zabezpieczeń, które w wielu przypadkach udaremniają 
wykorzystywanie tego typu podatności. Jak można się domyślić, znalazły 
się także bardzo ciekawe sposoby na ich obejście.
\subsection{Polityka W\string^X}
W poprzednim przykładzie zakładaliśmy, że nie ma podziału na dane i kod. 
W praktyce jednak, nie ma żadnego powodu, aby segmenty takie jak stos
były wykonywalne. Można też pójść jeszcze dalej i stwierdzić, że żadna 
sekcja zawierająca dane nie powinna być wykonywalna. Taka polityka 
nazywa się \textbf{Write XOR Execute}, co znaczy tyle, że pamięć nigdy 
nie powina być jednocześnie zapisywalna i wykonywalna. Wprowadzenie tej idei 
w życie może być wykonane na poziomie systemu operacyjnego, ale skutkowałoby 
to spadkiem wydajności\cite{bitnx}. Rozwiązaniem tego problemu było skorzystanie 
ze wsparcia sprzętowego. Producenci niektórych procesorów(np. tych z rodziny x86), 
zdecydowali się dodać do każdego wpisu w tablicy stron bit NX(No eXecute), który 
powoduje, że przy próbie wykonania kodu ze strony z uswawionym bitem NX zostanie 
wygenerowany wyjątek procesora. Polityka ta została wdrożona stosunkowo niedawno, 
w systemach Windows oraz Linux została ona zaimplementowana w roku 2004\cite{bitnx}.
Takie podejście ma jednak swoje wady. Niektóry technologie, takie jak np. 
kompilacja JIT, potrzebują możliwości jednoczesnego pisania i wykonywania 
fragmentów pamięci. Oczywiście, w Linuxie istnieje wywołanie systemowe 
\textbf{mprotect(2)}, pozwalające zmienić prawa dostępu do pamięci, ale to 
wprowadza narzut czasowy, a przecież celem kompilacji JIT jest zwiększenie wydajności.
\par
Obecnie, mechanizmy oddzielenia danych i kodu są bardzo powszechne(w szególności
na architekturze x86), więc raczej nie spotykamy się z przypadkami gdy jakaś strona
w pamięci jest jednocześnie zapisywalna i wykonywalna. Z tego powodu, ataki
polegające na wstrzykiwaniu kodu i wykonywania go, są niemożliwe, przez 
co trzeba szukać innych wektorów ataku.
\subsubsection{Atak ret2libc}
Wprowadzenie idei niewykonywalnej pamięci skutecznie powstrzymuje atakującego od 
wykonywania wstrzykniętego kodu. Można jednak zadać sobie pytanie, czy tylko kod 
umieszczony w pamięci przez atakującego może narobić szkód? Odpowiedź brzmi nie, i 
na tym opiera się atak \textbf{ret2libc}. Żeby zrozumieć na czym on polega, 
przenieśmy się na chwilę do przykładu z rozdziału pierwszego. Oczywiście nie ma 
tam żadnego złośliwego kodu, ale trzeba też wziąć pod uwagę jak będzie wyglądała 
pamięc wirtualna procesu, po uruchomieniu programu. Jest to program napisany w 
języku c, a więc do pamięci będzie załadowana biblioteka standardowa(czyli 
inaczej libc), w której znajduje się funkcja \textbf{system}, która używając 
powłoki, wywołuje podane jako argument polecenie. W szczególności, jeśli atakujący 
zdołałby wywołać tę funkcję z paramentem ``\textit{/bin/sh}'', uzyskałby dostęp 
do powłoki z prawami użytkownika, do którego należy dany proces, co może skonczyć 
się bardzo źle jeśli łączymy się z aplikacją przez sieć, a jeszcze gorzej jeśli 
właścicielem jest root. Nie porzucając założenia o znajomości całej przestrzenii 
adresowej procesu,możemy skoczyć do funkcji \textbf{system} tak samo jak skoczyliśmy 
na stos w przykładzie poprzednim, czyli nadpisując adres powrotu i wykorzystując 
asemblerową instrukcję ret. A co z argumentami? Przecież kontrolujemy stos, więc 
nie ma problemu abyśmy zaraz za nadpisanym adresem powrotu umieścili argumenty. 
Jeśli argumenty są przekazywane przez rejestry to sprawa się nieco komplikuje, 
ale można temu zaradzić stosując ROP, opisane w następnym rozdziale.
 
\subsection{ASLR}
Podczas ataku, bardzo często potrzebna jest znajomość adresów absolutnych. W 
poprzednich przykładach posługiwaliśmy się adresami stosu oraz segmentu kodu 
biblioteki standardowej języka C. Z tego powodu, jednym z mechanizmów mających na 
celu utrudnienie atakującemu wykorzystania błędów bezpieczeństwa jest randomizacja 
przestrzeni adresowej(ang. Addres space layout randomization). Ta idea wprowadza 
jednak pewne komplikacje, np. do poprawnego działania, kompilator nie może 
wygenerować kodu odnoszącego się do adresów absolutnych. Z tym problemem jednak 
można sobie poradzić generując kod PIC(ang. Position independent code), czyli kod, 
który posługuje się tylko adresami względnymi, działający niezależnie od tego 
gdzie zostanie załadowany. Skompilowanie kodu jako PIC nie gwarantuje jednak, 
że wszystkie segmenty będą ładowane losowo. Mechanizm ASLR jest zaimplementowany 
w jądrze systemu operacyjnego(w linuxie został wprowadzony w roku 2005). 
Niestety, w większości systemów(możliwe że nawet we wszystkich) rozkład 
prawdopodobieństwa nie jest jednostajny, powodem jest np. to, że adres pod 
którym ładowany jest segment musi być wyrównany do wielkości strony. Na 
losowość wpływa jeszcze wiele czynników, ale nie będziemy wchodzić w szczegóły. 
Ludzie przeprowadzili wiele testów pozwalających określić entropię ładowania 
poszczególnych segmentów, np. jak podaje \cite{aslr},w przypadku ładowania 
bibliotek dzielonych, na 32-bitowym systemie linux mamy tylko od 8 do 19 bitów, 
co znaczy mniej więcej tyle, że dysponując jedynie 32-bitową przestrzenią adresową, 
ASLR jest praktycznie bezużyteczny, ponieważ atak typu bruteforce to kwestia minut. 
Wersja 64-bitowa jest nieco lepsza, ponieważ entropia wacha się pomiędzy 28 a 35 
bitami(w zależności od implementacji), przez co w przypadku poprzednio podanego 
przykładu atakujący jest praktycznie bezradny.
\subsubsection{Return Oriented Programming}
Wiele kompilatorów generuje kod typu PIC domyślnie, co nie znaczy jednak, że zawsze 
i wszędzie będziemy spotykać tylko taki kod. Skutkiem tego jest to, że w losowych 
miejscach będą wszystkie segmenty oprócz kodu naszej aplikacji. Można zadać sobie 
pytanie, czy można to jakoś wykorzystać, nawet jeśli w kodzie naszej aplikacji 
nie ma tak krytycznych funkcji jak \textbf{system}? Odpowiedź brzmi tak. Używa 
się do tego bardzo ciekawej i ogólnej metody zwanej \textbf{Return Oriented 
Programming}(w skrócie \textbf{ROP}). Należy myśleć o niej jako o sposobie 
eksploitacji, a nie jak o konkretnym ataku. Załóżmy najpierw, że kontrolujemy 
adres powrotu z funkcji oraz kawałek stosu za nim. Idea polega na tym, by 
wykonywać kod po kawałku. Aby tego dokonać, będziemy używać tylko takich 
kawałków kodu asemblera które spełniają kilka warunków:
\begin{enumerate}
\item zawierają kilka instrukcji(np. od dwóch do pięciu)
\item nie używają instrukcji \emph{push}
\item kończą się instrukcją \emph{ret}
\end{enumerate}
Przyjęło się, że tego typu fragmenty kodu nazywa się \textbf{gadżetami}. Przykładowo:

\begin{flushleft}
pop rax\newline
pop rdx\newline
ret
\end{flushleft}

Teraz wystarczy adres powrotu nadpisać adresem gadżetu, za adresem powrotu wpisać 
adres kolejnego gadżetu, być może w międzyczasie wpisując na stos jakieś dane, tak 
by załadować je do rejestru instrukcją \emph{pop}. W przypadku programów zawierających 
duża ilość kodu bardzo łatwo będzie znaleźć taki ciąg gadżetów, który umieszcza w 
rejestrach odpowiednie wartości, a następnie używa instrukcji \emph{syscall}, 
by wykonać wywołanie systemowe \textbf{execve} z argumentem ``\textit{/bin/sh}''.

\subsection{Ciasteczka na stosie}
Dawniej, w kopalniach węgla problemem było ulatnianie się silnie toksycznego 
tlenku węgla. Aby zwiększyć bezpieczeństwo, w kopalniach umieszczano kanarki, 
które dzięki szybkiemu oddychaniu oraz szybkiemu metabolizmowi, w przypadku 
ulatniania się toksycznego gazu, zdychały bardzo szybko, co dawało górnikom czas 
na ucieczkę. Na podobny pomysł wpadli projektanci kompilatorów, którzy postanowili, 
że w ramce stosu, przed adresem powrotu, umieszczą dodatkową wartość wielkości słowa
maszynowego, często nazywaną ciasteczkiem bądź kanarkiem. Wartość ta jest całkowicie 
losowa, a przed powrotem z funkcji sprawdzane jest czy się jakkolwiek zmieniła, 
jeśli tak, program kończy swoje działanie w sposób tak bezpieczny, jak to tylko 
możliwe, a jeśli nie, program kontynuuje działanie. Wadą jest to że ogranicza 
wydajność, ponieważ dokleja do funkcji kod powodujący narzut czasowy. Z tego 
powodu kompilatory posługują się heurystykami dotyczącymi tego czy warto umieszczać
ciasteczko na stosie. Według \cite{cookie}, ciasteczko zostanie umieszczone na
stosie gdy funkcja spełnia choć jeden z następujących warunków:
\begin{itemize}
\item Operuje na tablicy, która jest większa niż cztery bajty, ma więcej niż dwa 
elementy i żaden z elementów nie jest wskaźnikiem.
\item Zawiera strukturę danych o rozmiarze większym niż 8 bajtów, 
niezawierającą żadnych wskaźników.
\item Rezerwuje pamięć na stosie przy użyciu funkcji \emph{alloca} lub 
\emph{\_malloca}
\item Obejmuje dowolną strukturę, lub klasę, która zawiera wymienione typy danych.
\end{itemize}
Oprócz tego, podczas kompilacji należy podać opcję \textit{-fstack-protector}, 
lub \textit{-fstack-protector-all} by wymusić używanie ciasteczek we wszystkich 
funkcjach. 
\par
Ciasteczek nie da się pokonać w żaden sprytny sposób, jednak to że 
znajdują się w okolicach kontrolowanego przez użytkownika bufora, czasami daje 
pewne możliwości by odkryć ich wartość. Gdyby w naszym przykładzie, zamiast
gets, użyto jakiejkolwiek innej funkcji powodującej buffer overflow, ale nie
umieszczającej bajtu zerowego na końcu, wystarczyłoby wpisać do bufora 
tyle bajtów, by funkcja puts razem z zawartością bufora, wypisała także ciasteczko. 
Co prawda nie uda nam się osiągnąć niczego więcej(potrzebna byłaby jeszcze 
jedna okazja do przepełnienia bufora), ale pokazuje to, że odkrycie wartości 
ciasteczka wcale nie musi być trudne. 
\par
Problemem przy implementacji tej technologii może być źródło entropii. 
O ile podczas normalnego użytkowania systemu możemy pozyskiwać entropię z różnych 
źródeł np. z szumu termicznego, to podczas bootowania systemu bedzie nieco 
trudniej, no bo przecież żeby skorzystać ze wsparcia sprzętowego, system 
operacyjny musi załadować sterowniki, czyli inaczej mówiąc musi wykonać jakiś kod. 
No ale skoro brak nam losowości to jak wygenerować losowe ciasteczka? Jak widać 
determinizm komputerów nie zawsze współgra z bezpieczeństwem, przykładowo, okazało 
się, że w wielu modułach systemu Windows XP, możliwe było odgadnięcie ciasteczka 
bez większego problemu\cite{cookieentropy}.
\subsection{Shadow Stack}
W 2015 roku pojawiła się idea wprowadzenia tzw. Shadow Stack\cite{shadow}, która ma 
raz na zawsze zakończyć wykorzystywanie błędów związanych z nadpisaniem adresu 
powrotu, co zazwyczaj jest spowodowane przepełnieniem bufora. Polega ona na tym 
by przy wywoływaniu funkcji, oprócz umieszczania adresu powrotu na zwykłym stosie,
wrzucić go także na specjalny stos, kontrolowany przez system operacyjny oraz 
hardware, a przy powrocie sprawdzać, czy adresy na obu stosach są sobie równe, 
jeśli nie, rzucany jest wyjątek. Niestety, wprowadza to kolejny narzut czasowy, 
w niektórych przypadkach dość duży. Mimo to, w czerwcu roku 2016, Intel wydał 
ponad 100-stronnicowy dokument zapowiadający wprowadzenie tej technologii w 
życie\cite{shadowintel}. Niektórzy\cite{riprop} nazywają tę technologię 
``\emph{RIP ROP}'', sugerując że to koniec eksploitacji tego typu. Jak 
będzie w rzeczywistości? Czy powstaną nowe techniki które zastąpią ROP? 
Czas pokaże, jest jeszcze za wcześnie by wydawać wyroki.
\subsection{Buffer Overflow na stercie}
Dotychczas mówiąc o przepełnieniu bufora zakładaliśmy, że wszystko dzieje się 
na stosie. A co jeśli  bufor znajduje się na stercie? Jest lepiej, gorzej? 
Na pewno jest źle. By dobrze przedstawić wszystkie konsekwencje, które za tym 
stoją wymagana jest wiedza o tym, jak działa alokator pamięci, czyli np. funkcja
\textbf{malloc} z języka C. Implementacji tej funkcji jest kilka, przykładowo, 
Linux używa algorytmu \textbf{ptmalloc}, który jest bardzo 
skomplikowany\cite{malloc}, zawierający ponad 6 tysięcy linii kodu w języku 
C dlatego nie jest możliwe jego szczegółowe omówienie.

\section{Buffer Overflow wiecznie żywy}
Mimo wyżej wymienionych zabezpieczeń podatności typu buffer overflow nadal 
występują i nadal są wykorzystywane. Nie jest to coś, co można zbagatelizować, 
ponieważ skutki tego typu podatności są katastrofalne, często prowadzi to do 
zdalnego wykonania kodu. W bazie CVE, czyli największym zbiorze wszystkich
znanych publicznie podatności, zazwyczaj dostają one stopień 
blisko 9, czyli bardzo wysoki, biorąc pod uwagę że jest to skala od 1 do 10,
gdzie 10 znaczy najwyższy stopień zagrożenia. 
Jak często występują takie podatności? Można się o tym samemu przekonać 
wchodząc na stronę \emph{https://www.cvedetails.com/}, wybierając najwyższy 
stopień zagrożenia(9-10), wciskając \textbf{CTRL}+\textbf{F} oraz wpisując 
``\emph{buffer overflow}''. W moim przypadku, tzn. w dniu 13.11.2017 o godzinie 
00:57 widać 5 różnych podatności, najstarsza z nich pochodzi z przed trzech 
miesięcy, a najmłodsza z przed piętnastu dni. Jak to możliwe? Myślę, że powodów 
jest kilka. Jednym z nich jest złożoność oprogramowania. Skoro w kilkulinijkowym 
programie można zostawić furtkę dla atakującego, to ile takich furtek można 
zostawić w oprogramowaniu zawierającego miliony linii kodu? Z praktycznego punktu 
widzenia, nie jest możliwe, aby tego typu oprogramowanie nie zawierało dziur, i 
nie chodzi tutaj o głupie pomyłki lub błędy wynikające z niewiedzy. Nad jądrem 
Linuxa pracowali wybitni programiści, mimo to, od początku jego istnienia na 
jaw wychodzą różne, mniej lub bardziej poważne błędy. Można wyobrazić sobie sytuację 
gdzie dwóch programistów pisze dwie różne funkcje, takie, że jedna zależy od 
drugiej, oraz każda z nich korzysta z bufora pochodzącego od użytkownika. Co, 
jeśli pierwszy programista założy, że sprawdzenie wielkości bufora powinno odbywać 
się wcześniej lub później, więc nie zaimplementuje tego, natomiast drugi programista 
założy, że wielkość bufora została sprawdzona przez programistę numer jeden? Obaj 
będą korzystać z potencjalnie niebezpiecznych danych. Zarządzanie pamięcią w 
dużych projektach nie jest rzeczą prostą, ponieważ tworzy wiele niepotrzebnych 
zależności, oraz zmusza programistów do skupiania się na niskopoziomowych detalach.
\par
Innym czynnikiem wpływającym na powstawanie błędów jest ludzka ignorancja. Wielu 
ludzi nie interesuje bezpieczeństwo, no bo co złego może się stać, jeśli nie 
użyje się funkcji kontrolującej ilość kopiowanych bajtów do bufora? ``\emph{Skoro 
sprawdziłem, że działa to działa, i nie ma się czym przejmować}''. Niestety 
ludzie którzy nigdy niczego umyślnie nie zepsuli często nie doceniają zagrożeń 
czychających na każdym kroku przy używaniu języków niskopoziomowych. Ciężko 
coś na to poradzić, szczegołnie że bezpieczeństwo niskopoziomowe jest dość 
trudną dziedziną, i nie można wymagać od ludzi by znali jej szczegóły. 
Z tego powodu odchodzi się od języków typu C, na rzecz takich, które nie dają 
programiście tak dużych możliwości. Nie wszystko jednak można napisać w Javie, 
w szczególności nie można np. napisać maszyny wirtualnej Javy, która także 
nie może być napisana w innym języku wysokopoziomowym ze względów wydajnościowych.
\par
Ogólnie rzecz biorąc, niektórzy próbują się zabezpieczać, inni próbują uciekać, 
i mimo że widać tego efekty, to zarówno buffer overflow jak i podobne, 
niskopoziomowe błędy mają się dobrze i dalej robią wiele szkód.

\begin{thebibliography}{99}
\bibitem{bitnx} Argento Daniele, Boschi Patrizio, Del Basso Luca, 
\textit{A hardware-enforced BOF protection}
\url{http://www.index-of.es/EBooks/NX-bit.pdf}

\bibitem{aslr} Hector Marco-Gisbert, Ismael Ripoll 
\textit{Exploiting Linux and PaX ASLR’s weaknesses on 32- and 64-bit systems}
\url{https://www.blackhat.com/docs/asia-16/materials/asia-16-Marco-Gisbert-Exploiting-Linux-And-PaX-ASLRS-Weaknesses-On-32-And-64-Bit-Systems.pdf}

\bibitem{cookie} Tomasz Kwiecień
\textit{Praktyczna Inżynieria Wsteczna - Rozdział 3}

\bibitem{cookieentropy} Matthew "j00ru" Jurczyk, Gynvael Coldwind
\textit{Windows Kernel-mode GS Cookies subverted}
\url{http://vexillium.org/dl.php?/Windows_Kernel-mode_GS_Cookies_subverted.pdf}

\bibitem{shadow} Thurston H.Y. Dang, Petros Maniatis, David Wagner 
\textit{The Performance Cost of Shadow Stacks and Stack Canaries}
\url{https://people.eecs.berkeley.edu/~daw/papers/shadow-asiaccs15.pdf}

\bibitem{shadowintel} Intel 
\textit{Control-flow Enforcement Technology Preview}
\url{https://software.intel.com/sites/default/files/managed/4d/2a/control-flow-enforcement-technology-preview.pdf}

\bibitem{riprop}  Chris Williams 
\textit{RIP ROP: Intel's cunning plot to kill stack-hopping exploits at CPU level}
\url{https://www.theregister.co.uk/2016/06/10/intel_control_flow_enforcement/}

\bibitem{malloc}  Dhaval Kapil
\textit{Heap Exploitation}
\url{https://heap-exploitation.dhavalkapil.com/}
\end{thebibliography}

\end{document}

