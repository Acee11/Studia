\documentclass[11pt,wide]{article}

\usepackage[T1]{fontenc}
\usepackage[utf8]{inputenc}
\usepackage{polski}
\usepackage{lmodern}
\usepackage{amsmath,amsthm}
\usepackage{algpseudocode}
\usepackage{pgfplots}
\usepackage{enumerate}
\newtheorem{thr}{Twierdzenie}
\newtheorem{lem}[thr]{Lemat}
\newtheorem*{dfn}{Definicja}
\usepackage{graphicx}
\graphicspath{ {img/} }
\usepackage[export]{adjustbox}
\title{Rozwiązanie zadania 3\\
}
\author{Wojciech Balik}


\begin{document}
\maketitle     
Załóżmy że mamy dane ciągi: \\
\begin{center}
$A = a_1 < a_2 < ... < a_n$ \\
$B = b_1 < b_2 < ... < b_n$ \\
\end{center}

Zdefiniujmy $2n$ ciągów w następujący sposób: \\
\begin{center}
$C_1 = a_1 < b_1 < a_2 < b_2 < ... < a_n < b_n$
\end{center}
(Czyli innymi słowy: $a_i < b_j \iff i \leq j$) \\
Pozostałe ciągi są tworzone poprzez zamiane miejscami elementów $a_i$ i $b_i$, lub $b_i$ oraz $a_{i+1}$ \\
Przez zamiane miejscami mam na myśli zamiene razem z relacją mniejszości, tzn.
\begin{center}
$C_2 = b_1 < a_1 < a_2 < b_2 < ... < a_n < b_n$ \\
$C_3 = a_1 < a_2 < b_1 < b_2 < ... < a_n < b_n$ \\
itd.
\end{center}

Na początku, algorytm ma wiedzę tylko o relacjach w ciągach $A$ oraz $B$, więc potencjalnym rozwiązaniem może być każdy ciąg $C_i$. \\
Gdyby algorytm zadał pytanie $a_2 <_? b_1$ a adwersarz odpowiedziałby "tak", to zostałoby jedno potencjalne rozwiązanie - $C_3$, 
jednak my przyjmiemy inną strategię. Na pytanie $a_i <_? b_j$ odpowiadamy "tak" $\iff i \leq j$.

\begin{lem}
Niech $L_k$ oznacza ilość potencjalnych rozwiązań po $k$ porównaniach. Wówczas:
\begin{center}
$L_k \geq 2n - k$
\end{center}
\end{lem}
\begin{proof}
Dowód przez indukcję względem $k$. \\
Teza jest przwdziwa dla $k = 0$ (mamy $L_0 = 2n$). \\
Załóżmy że $T_{k-1} \geq 2n - k + 1$ i rozważmy przypadki(zakładam że algorytm nie pyta o coś co już wie):
\begin{enumerate}
\item 
Algorytm porównuje $a_i$ i $b_i$ lub $b_i$ i $a_{i+1}$, odpowiadamy zgodnie ze strategią. Jest to prawda dla wszystkich ciągów $C_i$ oprócz tego w którym zamieniliśmy te elementy miejscami, 
wówczas ten ciąg nie będzie więcej potencjalnym rozwiązaniem, a więc $T_k = T_{k-1}-1 \geq 2n-k$
\item 
Algorytm porównuje $a_i$ i $b_j$, gdzie $i \neq j$ oraz $i \neq j+1$ - odpowiadamy zgodnie ze strategią. Jest to prawda dla każdego ciągu $C_i$ a więc $T_k = T_{k-1} \geq 2n - k$.
\end{enumerate} 
\end{proof}
Na końcu działania algorytmu $L_k = 1 \geq 2n - k$, a więc $k \geq 2n-1$. 
\end{document}